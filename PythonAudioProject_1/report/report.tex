\documentclass{article}

\begin{document}

\title{Project Report: Random Audio Player with GUI}
\author{Parth Kansagra}
\date{May 16,2023}

\maketitle

\section{Introduction}
The goal of this project was to create a Python program that plays audio files from a specified folder in a random order, with the ability to navigate to the next and previous tracks using a graphical user interface (GUI). The program utilizes the \texttt{pygame} library for audio playback and the \texttt{tkinter} library for creating the GUI.

\section{Implementation}
The program is implemented in Python and consists of the following key components:

\begin{itemize}
  \item File Selection: The user provides the path to the folder containing the audio files.
  \item Randomization: The program lists all the audio files in the folder and shuffles them randomly.
  \item Audio Playback: The \texttt{pygame} library is used to load and play the audio files. The program sets the volume level and plays the files sequentially.
  \item GUI Creation: The \texttt{tkinter} library is used to create a GUI window with "Next" and "Previous" buttons.
  \item Button Functionality: The "Next" button increments the current index, stops the current audio playback, and starts playing the next track. The "Previous" button decrements the index and plays the previous track. Wrapping around is handled to ensure seamless navigation between tracks.
\end{itemize}

\section{Usage}
To use the program, follow these steps:
\begin{enumerate}
  \item Run the program in a Python environment (Python 3 or above).
  \item Provide the path to the folder containing the audio files.
  \item The program will display a GUI window with "Next" and "Previous" buttons.
  \item Click the "Next" button to go to the next track and the "Previous" button to go to the previous track.
  \item The program will play the audio files in a random order each time it is run.
\end{enumerate}

\section{Dependencies}
The program relies on the following external libraries:
\begin{itemize}
  \item \texttt{pygame}: Used for audio playback and volume control.
  \item \texttt{tkinter}: Used for creating the GUI window and buttons.
\end{itemize}

Ensure that these libraries are installed in the Python environment before running the program.

\section{Conclusion}
The project successfully achieves its objective of creating a random audio player with GUI functionality. The program allows users to enjoy their audio collection in a randomized order and provides convenient navigation options. The combination of \texttt{pygame} and \texttt{tkinter} libraries provides a seamless and intuitive user experience.

In summary, the random audio player project demonstrates the effective utilization of Python libraries to create an interactive and enjoyable audio playback experience.

\end{document}

\documentclass{article}

\begin{document}

\title{Project Report: Random Audio Player with GUI}
\author{Your Name}
\date{\today}

\maketitle

\section{Introduction}
The goal of this project was to create a Python program that plays audio files from a specified folder in a random order, with the ability to navigate to the next and previous tracks using a graphical user interface (GUI). The program utilizes the \texttt{pygame} library for audio playback and the \texttt{tkinter} library for creating the GUI.

\section{Implementation}
The program is implemented in Python and consists of the following key components:

\begin{itemize}
  \item File Selection: The user provides the path to the folder containing the audio files.
  \item Randomization: The program lists all the audio files in the folder and shuffles them randomly.
  \item Audio Playback: The \texttt{pygame} library is used to load and play the audio files. The program sets the volume level and plays the files sequentially.
  \item GUI Creation: The \texttt{tkinter} library is used to create a GUI window with "Next" and "Previous" buttons.
  \item Button Functionality: The "Next" button increments the current index, stops the current audio playback, and starts playing the next track. The "Previous" button decrements the index and plays the previous track. Wrapping around is handled to ensure seamless navigation between tracks.
\end{itemize}

\section{Usage}
To use the program, follow these steps:
\begin{enumerate}
  \item Run the program in a Python environment (Python 3 or above).
  \item Provide the path to the folder containing the audio files.
  \item The program will display a GUI window with "Next" and "Previous" buttons.
  \item Click the "Next" button to go to the next track and the "Previous" button to go to the previous track.
  \item The program will play the audio files in a random order each time it is run.
\end{enumerate}

\section{Dependencies}
The program relies on the following external libraries:
\begin{itemize}
  \item \texttt{pygame}: Used for audio playback and volume control.
  \item \texttt{tkinter}: Used for creating the GUI window and buttons.
\end{itemize}

Ensure that these libraries are installed in the Python environment before running the program.

\section{Conclusion}
The project successfully achieves its objective of creating a random audio player with GUI functionality. The program allows users to enjoy their audio collection in a randomized order and provides convenient navigation options. The combination of \texttt{pygame} and \texttt{tkinter} libraries provides a seamless and intuitive user experience.

Further improvements could include adding additional features such as a progress bar, a playlist view, or the ability to loop or shuffle the tracks on demand. Additionally, incorporating error handling and validation for file formats could enhance the overall robustness of the program.

In summary, the random audio player project demonstrates the effective utilization of Python libraries to create an interactive and enjoyable audio playback experience.

\end{document}

